\section{Market Opportunities: Abandoned Mine Shafts}

    \subsection{Market Opportunity Description}

        This market opportunity focuses on repurposing deep abandoned mine shafts for gravity-based energy storage (GBES) systems. Instead of constructing entirely new buildings as that is more expensive. Existing vertical shafts are used to lift and lower heavy masses, converting electrical energy to potential energy and back.  

        Target customers for this market include:  

        \begin{itemize}
            \item Utilities seeking large-scale energy storage  
            \item Regions with high renewable penetration requiring grid flexibility  
            \item Operators of abandoned mine lands interested in energy reuse  
        \end{itemize}  

        This opportunity is supported by trends including the growing need for long-duration energy storage, the ab undance of abandoned mine shafts and incentives for energy storage that isn't bad for the environment.  

    \subsection{Output and Feasibility}
    
        Since the amount of energy stored in gravity-based energy storage (GBES) is directly correlated by how heavy the weight is, and how far it is raised. Repurposing deep, abandoned mine shaft for GBES would be a smart choice. As mine shafts regulatory go over 1km down this would be equivalent of building a 1km building above ground, which is very expensive to build.
        
        The best parts is that many regions have hundreds of thousands of dormant mine shafts roughly 550,000 in the USA with depths often exceeding one kilometer\cite{RTBC2024}. These mines offer large vertical drop and existing infrastructure (hoists, grid connections), greatly reducing civil construction costs\cite{RTBC2024}\cite{Hunt2023}.
    
        The result is very high potential energy storage per site with minimal work. There were some IIASA researchers that said the following about mineshafts for GBES: "the deeper and broader the mineshaft, the more power can be extracted..., and the larger the mine, the higher the plant's energy storage capacity".\cite{Hunt2023}.
    
        The stored energy in a GBES system is given by the potential energy $E$ of the mass $m$ raised by height $h$: $E = m\cdot g\cdot h$ where $g$ is approximately $9.81 {m\over s^2}$.

        For example, a $3000$ tonne ($3\dot 10^6$kg) weight lifted $1000$ meters would result in $E\approx 3\cdot10^6\cdot9.81\cdot1000\approx 2.94\cdot10^10$ joules, or about $8.2$ MWh. Even a $1000$ tonne mass over $1 km$ yields about $2.7\dot106$ joules, or roughly $0.75$ MWh.

    \subsection{Attractiveness Assessment (MON Worksheet 2)}

        \subsubsection{Potential}

            \paragraph{Compelling Reason to Buy — High} 
            
                GBES does not use harmfull chemicals, and have zero emissions while in use. They are durable and long lasting, meaning it's not wastefull to build.

            \paragraph{Market Volume — Mid}  
            
                Tens of thousands of abandoned shafts are available globally (for example 500,000 in the USA\cite{TheWeek}), but only a fraction are suitable for GBES due to location, depth, or condition.  

            \paragraph{Economic Viability — Mid–High} 
            
                Repurposing shafts costs less in construction costs compared to new infrastructure, as building buildings that tall would be very expenive.
        
        \subsubsection{Challenges}

            \paragraph{Implementation Obstacles — High}  
            
                Regulatory approvals, safety inspections, and environmental remediation are required. Engineering challenges include shaft stabilization, water ingress management, and installation of heavy lifting systems.  

            \paragraph{Time to Revenue — High}
            
                Permitting, safety verification, and equipment installation can take several years before a system becomes operational.  

            \paragraph{External Risks — Mid–High}  
            
                Competition exists from batteries, pumped hydro, and thermal storage. Public acceptance, insurance, and liability concerns may limit adoption.  

        \subsubsection{Summary Ratings}

            \begin{itemize}
                \item Potential: High  
                \item Challenge: High  
                \item Economic viability: Mid–High  
            \end{itemize}  
            
            Overall attractiveness: Large-scale, high-potential opportunity with significant technical and regulatory risk  
            
    \subsection{SWOT Analysis}

        \subsubsection{Strengths}

            \begin{itemize}
                \item High energy storage potential per site  
                \item Long operational lifetime  
                \item Leverages existing infrastructure  
                \item Environmentally safer than chemical batteries  
            \end{itemize}
            
        \subsubsection{Weaknesses}

            \begin{itemize}
                \item Requires detailed engineering assessment  
                \item Regulatory and permitting complexity  
                \item Site-specific feasibility limits market size  
            \end{itemize}

        \subsubsection{Opportunities}

            \begin{itemize}
                \item Renewable energy integration  
                \item Redevelopment of brownfield or mine lands  
                \item Ancillary services for electricity grids  
            \end{itemize}

        \subsubsection{Threats}

            \begin{itemize}
                \item Competing energy storage solutions (batteries, pumped hydro)  
                \item Liability and insurance challenges  
                \item Environmental remediation requirements  
            \end{itemize}

    \subsection{Interview Insights}

        I've not had an interview related to this market opportunity, we do have contacts with someone who works in mines. This interview is at the moment, not yet planned.
