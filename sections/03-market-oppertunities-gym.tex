\section{Market Opportunity Evaluation: Gym \& Fitness Centers} 
    
    \subsection{Market Opportunity Description} 
    
        This market opportunity explores the integration of a \textbf{gravity–based energy and engagement system} into commercial gyms and university sports facilities. The idea is simple: as people work out, their effort lifts a mass vertically, storing gravitational potential energy. The rising weight gives users a clear, physical indicator of how much work they have done. Later, this stored energy can be converted into small amounts of electricity for low–power uses inside the gym. 
        
        Our target customers include: 
        
        \begin{itemize} 
            \item Medium to large commercial gyms aiming to stand out with innovative experiences. 
            \item University sports centres looking for sustainability features that engage students. 
            \item Corporate and institutional fitness centres with ESG or green–building goals. 
        \end{itemize} 
        
        Two trends strongly support this opportunity: the growth of \emph{experiential fitness} (gamified and interactive workouts) and the increasing emphasis on \emph{sustainability} as a differentiator for gyms and sports clubs. 
        
    \subsection{Technical Feasibility and Energy Estimate} 
    
        A natural first question gym owners will ask is: \emph{How much energy does this system actually produce?} Although the system is not meant to power the entire facility, having a realistic estimate helps position the product correctly. 

        Consider lifting a mass of $m = 100\ \mathrm{kg}$ by $h = 2\ \mathrm{m}$. The energy stored is: 
        \[ E = mgh = 100 \times 9.81 \times 2 = 1962\ \mathrm{J}. \] 
        
        Converting this to kilowatt–hours gives: 
        \[ E_{\mathrm{kWh}} = \frac{1962}{3.6 \times 10^{6}} \approx 0.000545\ \mathrm{kWh}. \] 
        
        \textbf{Interpretation (100 kg):} one full lift of a 100\,kg mass to 2\,m stores just over $1900$ joules, which is about $0.000545$ kWh. Even if 100 gym members each produced this amount daily, the total would be: 
        \[ 100 \times 0.000545\ \mathrm{kWh} = 0.0545\ \mathrm{kWh/day}. \] 
        
        We now scale this to lifting \textbf{1 metric ton}, that is $m = 1000\ \mathrm{kg}$ over the same height $h = 2\ \mathrm{m}$. The stored energy becomes: 
        \[ E = 1000 \times 9.81 \times 2 = 19620\ \mathrm{J}, \] 
        
        which in kilowatt–hours is: 
        \[ E_{\mathrm{kWh}} = \frac{19620}{3.6 \times 10^{6}} \approx 0.00545\ \mathrm{kWh}. \] 
        
        \textbf{Interpretation (1 ton):} lifting 1 ton to 2\,m stores about $0.00545$ kWh --- still less than 1\% of a typical smartphone battery (5--10 Wh). Even if 100 people performed this lift every day, the total energy would be: 
        \[ 100 \times 0.00545\ \mathrm{kWh} = 0.545\ \mathrm{kWh/day}. \] 
        
        \textbf{Summary:} These values clearly show that, even at extreme loads, the mechanical energy humans can generate is very small compared to real electricity use. Lifting 100 kg produces only $0.000545$ kWh, and lifting 1 ton produces just $0.00545$ kWh. This amount of energy cannot meaningfully power a gym; however, it \emph{can} power motivation. The true value of the system lies in creating a \textbf{green gym identity}, engaging users, and making their effort visible and rewarding—not in substantial electricity production. 
        
    \subsection{Attractiveness Assessment (MON Worksheet 2)} 

        \subsubsection{Potential} 
            
            \paragraph{Compelling Reason to Buy — \textit{High}} 
            
                User interviews highlight a consistent frustration: it is hard to \emph{see} progress during many strength workouts. The rising weight solves this directly and intuitively. Gym owners also value equipment that boosts retention and offers a unique experience. Combined with the sustainability aspect, this creates a convincing reason to buy. 
    
            \paragraph{Market Volume — \textit{Mid--High}} 
            
                The global fitness industry is huge, but not every gym is the ideal customer. Early adopters are likely boutique gyms, university facilities, and premium centres that actively seek differentiation. While the addressable market is narrower than the overall gym market, it is still large enough to support pilot deployments and scaling. 
            
            \paragraph{Economic Viability — \textit{Mid}} 
            
                Potential revenue streams include system sales, installation fees, and maintenance contracts. Sustainability grants may support early pilots. Although the mechanical system has relatively low wear, the required safety measures and custom installations increase initial costs. The economics appear viable but require carefully chosen pilot sites. 
            
        \subsubsection{Challenge} 
        
            \paragraph{Implementation Obstacles — \textit{High}} 
            
                The main barriers are physical: ceilings, floor anchoring, retrofitting difficulties, and safety regulations. Each installation needs some customization. Meeting certification requirements for public, high–traffic spaces is also a challenge. 

            \paragraph{Time to Revenue — \textit{Mid--High}} 
            
                Developing, testing, and certifying the system will require several iterations. Gym procurement cycles are typically slow, especially for equipment that requires installation work. A pilot–driven approach helps but does not eliminate the relatively long time to meaningful revenue. 
            
            \paragraph{External Risks — \textit{Mid}} 
            
                There are competing green–gym technologies like regenerative bikes or energy–harvesting floors. Supply chain risks exist for mechanical parts, and insurance requirements might add constraints. There is also a perception risk: if energy output is marketed too strongly, users may feel misled. Clear communication is essential. 

        \subsubsection{Summary Ratings} 
        
            \begin{itemize} 
                \item Potential: \textbf{High} 
                \item Challenge: \textbf{Mid--High} 
                \item Economic viability: \textbf{Mid} 
                \item Overall attractiveness: \textbf{Promising for targeted pilots} 
            \end{itemize} 

    \subsection{SWOT Analysis} 
    
        \subsubsection{Strengths} 
        
            \begin{itemize} 
                \item Very strong visual and motivational impact. 
                \item Clear sustainability value for marketing and branding. 
                \item Simple mechanical system with relatively low wear. 
            \end{itemize} 

        \subsubsection{Weaknesses} 
        
            \begin{itemize} 
                \item Low energy output. 
                \item Significant installation and space requirements. 
                \item Higher responsibility due to safety compliance. 
            \end{itemize} 

        \subsubsection{Opportunities} 
        
            \begin{itemize} 
                \item University and research pilots. 
                \item Boutique gyms looking for unique concepts. 
                \item Sustainability funding opportunities. 
            \end{itemize} 

        \subsubsection{Threats} 
            
            \begin{itemize} 
                \item Competing green fitness technologies. 
                \item Negative perception if marketed primarily as an energy–saving device. 
                \item Installation delays or supply chain issues. 
            \end{itemize} 
    
    \subsection{Interview Plan} 
        
        To strengthen our evidence base, we plan to: 
        
        \begin{itemize} 
            \item Conduct 8–12 additional user interviews across different gym types. 
            \item Interview couple gym managers or owners to understand purchasing behaviour and constraints. 
            \item Launch a short online survey for members of the pilot gym. 
        \end{itemize} 

    \subsection{Risks and Mitigations} 
    
        \paragraph{Low perceived energy yield.}
            
            We will frame the system as an engagement tool first, with energy production as a bonus. 
        
        \paragraph{Space and retrofit issues.} 
            
            We will target early adopters with available space and design modular installation options. 
        
        \paragraph{Safety concerns.}
            
            The system will include multiple safety layers and will undergo independent certification. 
    
    \subsection{Conclusion} 
    
        The gym and fitness market presents a promising opportunity for the gravity–based system. While the electrical output remains modest, the system's impact on engagement and sustainability storytelling gives it a strong value proposition. A focused pilot strategy will allow us to validate user motivation, refine the technical design, and assess willingness to pay. If pilot results are positive, the opportunity can be scaled into niche but meaningful segments of the fitness market. 
