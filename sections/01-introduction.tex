\section{Introduction}

    \subsection{What is the assignment?}

        In a world where we are increasingly more dependent on energy consumption, we must find ways to do that more effectively. But next to this, the degradation of our environment is always looming over all our choices. These factors combined bring us to the issue at hand: how can we store our energy in a safe, environmentally friendly way, while also cost-effectively addressing our energy needs? 
        
        This report will dive deeper into our market analysis, firstly describing which technology we chose to focus on and our reasoning behind it. Afterwards, we will describe our research into what market opportunities are available to us. We will go into detail about 5 of these opportunities and finally conclude which one we will proceed with.  

    \subsection{Our brainstorm for technologies}

        To start this entrepreneurial venture, we needed to decide which technology we would focus on. Instead of choosing one of the provided clients, we decided to do our own research into what interested us the most. From this research, we found three interesting options. 
         
        Firstly, we investigated Compressed Air Energy Storage (CAES). This technology involves compressing air in an underground cavern using excess energy. Whenever energy needs to be released, the pressure is lifted and energy will be available for external use. This method can be scaled to large capacities and is usable for long-duration storage. However, an underground cavern must be excavated to facilitate the storage. It is also highly mechanically complex, and the high pressure poses a risk of explosion. 
        
        Secondly, we researched Heat Exchange Batteries. This technology can store the energy as heat rather than electricity. The surplus electricity is turned into thermal energy and stored in materials like molten salt or stones. This heat can be used back to heat or it can be transformed back into electricity. The high market potential, in particular, industry and the built environment is the primary strength of the technology. Yet, the biggest drawback is that one of the largest initial markets, which is the data centers, has already largely acquired alternative solutions, which constrains the growth opportunities in that market. 

        Lastly, Gravity Based Energy Storage was explored where the energy is stored in this technology using electric power to raise heavy weights. In case of energy requirement, the masses are reduced again and produce electricity via generators. The key benefits associated with this technology are that it is visible, long-lasting, scalable, and cost-effective in the long term because of low-maintenance and high lifetime. Nevertheless, the system is also bulky and hard to fit into any existing space, as most of the infrastructure was not meant to hold huge moving masses. 

    \subsection{Chosen technology and the reasoning}
    
        The most promising option after comparison of the three technologies is Gravity Based Energy Storage, and hence, we decided to explore it deeper. To make this decision, we evaluated the advantages and disadvantages of each technology and compared them to the criteria we considered most important: sustainability, safety, lifespan, scalability, and long-term economic performance.
        
        The main advantages of gravity-based energy storage were: it works at almost zero emissions, as there is no need for chemical reactions or fuel combustion; it can be deployed anywhere, on industrial sites or renewable energy farms, even within urban infrastructure. Another important strength of this concept is the long mechanical life: indeed, mechanical systems degrade much more slowly than chemical batteries, and such a feature makes this technology competitive in the long run. Another strength is that the system offers very high safety with no fire and leakage risks, unlike lithium-ion batteries. Finally, gravity-based storage facilities are produced to make use of steel and concrete materials in ways that allow easy recyclability at the end of the product life cycle, thus engendering additional revenue from material recovery.
 
    \subsection{Applications from brainstorm and the MON worksheet}

        To identify which market opportunities can be developed with our technology, we, as a group, performed a brainstorming session. To guide this process, we utilized MON Worksheet 1. This worksheet helped us identify which are the major abilities of a gravity-based energy technology. Based on our discussion, we concluded that the technology has several key abilities: low degradation and long-lasting performance, scalability through adding more weight, a visual and easily understandable energy level, and minimal environmental impact due to almost no CO₂ emissions during operation. With our identified abilities as a guide, we were able to create several market opportunities.
    
    \subsection{Applications from prompt}

        To expand our initial brainstorming results, we looked further into detail for finding new market opportunity ideas with the use of the prompt described in Section II. Microsoft Copilot. To get the result we wanted, we asked the AI to generate a wide range of applications for gravity-based energy storage, including both well-known market segments and more novel or niche opportunities. The ideas generated by Copilot helped us look beyond our initial ideas and consider markets we had not discussed yet. After we got the generated results, we compared them with the market opportunities we thought of and later assessed them according to the criteria given in MON Worksheet 2.
    
    