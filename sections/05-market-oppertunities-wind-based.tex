\section{Gravity-Based Energy Storage for Solar \& Wind Energy Farms} 

    With an increased demand for renewable energy sources, such as wind and solar, comes an even more dire need for effective energy storage methods. This is because these before-mentioned technologies cannot guarantee 24/7 uptime, because they are dependent on weather conditions. In order to not waste any excess energy and ensure permanent energy uptime, we must look into what technology is most applicable. 

    At the moment, solar and wind farms use a couple different technologies to store their excess energy. Firstly, we look into the most common method, which is lithium-ion batteries. Their most prominent features include high energy density, long cycle life, and fast charging . They are also easy to add to a system, as they require limited space and are not dependent on any environmental factors. However, they are not flawless as they come with high cost, limited lifespan, safety concerns, environmental impact, and temperature sensitivity\cite{llcBT}.
    
    One technology that is specifically applied to solar farms is Thermal Energy Storage (TES). By using excess energy to heat up a material, energy can be stored in the form of heat. Heat is usually a waste by product in many processes, so it is a great way to make use of otherwise wasted energy. It also provides high energy density. However, its main drawbacks consist of safety risks, sensitivity to environmental factors and most of all, losses in dissipating heat during storage and use\cite{Rekioua2023}.

    \subsection{Market Opportunity Description}
        
        This is where Gravity Based Energy Storage comes in. It can bypass environmental limitations, due to its principle not requiring any. It is also much safer and environmentally friendly, as it does not involve any rare minerals or possibly dangerous chemicals. The energy can be stored with minimal losses as it will never dissipate in the form of heat. 
        
        For this particular application, large scale storage facilities will be built, such as the facility located in Rudong, China, built by Energy Vault\cite{EnergyVault}.  These projects will be mostly marketed towards local governments for municipalities, or corporations that manage large-scale energy farms.
        
        This application has been analysed using MON worksheet 2. The results can be found below. Potential 

    \subsection{Attractiveness Assessment (MON Worksheet 2)}

        \subsubsection{Potential}
        
            \paragraph{Compelling reason to buy: Super high}
            
                The need for a sustainable, large scale energy storage technology is high, and especially one that is as safe as GES is currently an unmet need. It is effective in solving the existing problems and it is better than current solutions in both storing energy efficiently and with minimal risk. 
            
            \paragraph{Market volume: High}
                
                The current market size, which consists of local governments, is quite good already, and the amount of solar- and wind farms that are being built is ever increasing. The need for energy storage will continue growing. 
            
            \paragraph{Economic viability: Mid}
                
                The margins and customers ability are lower as the facilities require large investments and governments are on a tight budget. Large scale corporations have more leeway however. 
        
        \subsubsection{Challenge}
            
            \paragraph{Implementation obstacles: Mid}
                
                As mentioned above, the investment is quite large and it will require a lenghty construction period. However, the materials are not hard to acquire, and no distribution is required. 
            
            \paragraph{Time to revenue: High}
            
                The sale cycle is incredibly long, as it takes years for the facility to be fully operational. Because of the initial investment, it will take multiple years for the facility to return its investment. 
            
            \paragraph{External risks: Low}
                
                There are little to no external risks, as it is not a saturated market, there are few third-party dependencies outside of construction companies, and due the elimination of environmental dependability, the systems can be built anywhere. 
            
                What follows from this analysis is that the overall potential of this market opportunity is high, while carrying medium risks. This makes it a viable option overall that, when overcoming the risks, can ensure long-term revenue.

    \subsection{Interview Insights}

        For this particular market opportunity, we have not conducted an interview yet. However, we have contact with a person within our network who works on solar panel installation, and we reached out to Energy Vault for more questions regarding their business.
