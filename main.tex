\documentclass{article}
\usepackage{graphicx}
\usepackage{geometry}
\geometry{margin=1in}
\usepackage{longtable}

\title{Market opportunities: identification and evaluation}
\author{Team Sissyphus}
\date{December 2025}

\begin{document}
    \maketitle
    \newpage
    \section{Introduction}

    \subsection{What is the assignment?}

        In a world where we are increasingly more dependent on energy consumption, we must find ways to do that more effectively. But next to this, the degradation of our environment is always looming over all our choices. These factors combined bring us to the issue at hand: how can we store our energy in a safe, environmentally friendly way, while also cost-effectively addressing our energy needs? 
        
        This report will dive deeper into our market analysis, firstly describing which technology we chose to focus on and our reasoning behind it. Afterwards, we will describe our research into what market opportunities are available to us. We will go into detail about 5 of these opportunities and finally conclude which one we will proceed with.  

    \subsection{Our brainstorm for technologies}

        To start this entrepreneurial venture, we needed to decide which technology we would focus on. Instead of choosing one of the provided clients, we decided to do our own research into what interested us the most. From this research, we found three interesting options. 
         
        Firstly, we investigated Compressed Air Energy Storage (CAES). This technology involves compressing air in an underground cavern using excess energy. Whenever energy needs to be released, the pressure is lifted and energy will be available for external use. This method can be scaled to large capacities and is usable for long-duration storage. However, an underground cavern must be excavated to facilitate the storage. It is also highly mechanically complex, and the high pressure poses a risk of explosion. 
        
        Secondly, we researched Heat Exchange Batteries. This technology can store the energy as heat rather than electricity. The surplus electricity is turned into thermal energy and stored in materials like molten salt or stones. This heat can be used back to heat or it can be transformed back into electricity. The high market potential, in particular, industry and the built environment is the primary strength of the technology. Yet, the biggest drawback is that one of the largest initial markets, which is the data centers, has already largely acquired alternative solutions, which constrains the growth opportunities in that market. 

        Lastly, Gravity Based Energy Storage was explored where the energy is stored in this technology using electric power to raise heavy weights. In case of energy requirement, the masses are reduced again and produce electricity via generators. The key benefits associated with this technology are that it is visible, long-lasting, scalable, and cost-effective in the long term because of low-maintenance and high lifetime. Nevertheless, the system is also bulky and hard to fit into any existing space, as most of the infrastructure was not meant to hold huge moving masses. 

    \subsection{Chosen technology + reasoning}
        
        ...
 
    \subsection{Applications from brainstorm + mon worksheet}

        ...
    
    \subsection{Applications from prompt}

        ...
    
    \subsection{Online research for applications}

        ...
    
    \section{Microsoft Copilot}
    
    For the assignment, we had to create and use a Microsoft Copilot prompt, and this is what we came up with: 
    
    I want to start with an entrepreneurial endeavor involving gravity base energy storage technology. The focus is on sustainability and the specific application of energy storage. I need to find several market opportunities that involve this technology, and I want the different market opportunities to follow from [MON worksheet 1], and afterwards I want them analyzed using [MON worksheet 2]. Give a clear table of these opportunities following these rules: 
    \begin{itemize}
        \item Include both more well-known market segments, as well as novel and niche segments;
        \item Use variety across industries;
        \item Use reliable sources such as industry reports, patents, and technology trend databases;
        \item Name both the market needs, as well as how this opportunity satisfies these needs;
        \item Take relevance into account.
    \end{itemize}

    The responds we got from that prompt was very detailed, which is why it’s in the appendix. We then proceeded to choose the market opportunities we thought had the most potential. 

%%%%%%%%%%%%%%%%%%%%%%%%%%%%%%%%%%%%%%%%%%%%%%%%%%%%%%%%%%%%%
%%%%%%%%%%%%%%%   PART 1 — GYM & FITNESS   %%%%%%%%%%%%%%%%%%
%%%%%%%%%%%%%%%%%%%%%%%%%%%%%%%%%%%%%%%%%%%%%%%%%%%%%%%%%%%%%

\section{Market Opportunity Evaluation: Gym \& Fitness Centers}

\subsection{Market Opportunity Description}

This market opportunity explores the integration of a \textbf{gravity–based energy and engagement system} into commercial gyms and university sports facilities. The idea is simple: as people work out, their effort lifts a mass vertically, storing gravitational potential energy. The rising weight gives users a clear, physical indicator of how much work they have done. Later, this stored energy can be converted into small amounts of electricity for low–power uses inside the gym.

Our target customers include:
\begin{itemize}
  \item Medium to large commercial gyms aiming to stand out with innovative experiences.
  \item University sports centres looking for sustainability features that engage students.
  \item Corporate and institutional fitness centres with ESG or green–building goals.
\end{itemize}

Two trends strongly support this opportunity: the growth of \emph{experiential fitness} (gamified and interactive workouts) and the increasing emphasis on \emph{sustainability} as a differentiator for gyms and sports clubs.

\subsection{Technical Feasibility and Energy Estimate}

A natural first question gym owners will ask is: \emph{How much energy does this system actually produce?} Although the system is not meant to power the entire facility, having a realistic estimate helps position the product correctly.

Consider lifting a mass of $m = 100\ \mathrm{kg}$ by $h = 2\ \mathrm{m}$. The energy stored is:
\[
E = mgh = 100 \times 9.81 \times 2 = 1962\ \mathrm{J}.
\]
Converting this to kilowatt–hours gives:
\[
E_{\mathrm{kWh}} = \frac{1962}{3.6 \times 10^{6}} \approx 0.000545\ \mathrm{kWh}.
\]

\textbf{Interpretation (100 kg):} one full lift of a 100\,kg mass to 2\,m stores just over $1900$ joules, which is about $0.000545$ kWh. Even if 100 gym members each produced this amount daily, the total would be:
\[
100 \times 0.000545\ \mathrm{kWh} = 0.0545\ \mathrm{kWh/day}.
\]

We now scale this to lifting \textbf{1 metric ton}, that is $m = 1000\ \mathrm{kg}$ over the same height $h = 2\ \mathrm{m}$. The stored energy becomes:
\[
E = 1000 \times 9.81 \times 2 = 19620\ \mathrm{J},
\]
which in kilowatt–hours is:
\[
E_{\mathrm{kWh}} = \frac{19620}{3.6 \times 10^{6}} \approx 0.00545\ \mathrm{kWh}.
\]

\textbf{Interpretation (1 ton):} lifting 1 ton to 2\,m stores about $0.00545$ kWh --- still less than 1\% of a typical smartphone battery (5--10 Wh). Even if 100 people performed this lift every day, the total energy would be:
\[
100 \times 0.00545\ \mathrm{kWh} = 0.545\ \mathrm{kWh/day}.
\]

\textbf{Summary:} These values clearly show that, even at extreme loads, the mechanical energy humans can generate is very small compared to real electricity use. Lifting 100 kg produces only $0.000545$ kWh, and lifting 1 ton produces just $0.00545$ kWh. This amount of energy cannot meaningfully power a gym; however, it \emph{can} power motivation. The true value of the system lies in creating a \textbf{green gym identity}, engaging users, and making their effort visible and rewarding—not in substantial electricity production.

\subsection{Attractiveness Assessment (MON Worksheet 2)}

\subsubsection{Potential}

\paragraph{Compelling Reason to Buy — \textit{High}.}
User interviews highlight a consistent frustration: it is hard to \emph{see} progress during many strength workouts. The rising weight solves this directly and intuitively. Gym owners also value equipment that boosts retention and offers a unique experience. Combined with the sustainability aspect, this creates a convincing reason to buy.

\paragraph{Market Volume — \textit{Mid--High}.}
The global fitness industry is huge, but not every gym is the ideal customer. Early adopters are likely boutique gyms, university facilities, and premium centres that actively seek differentiation. While the addressable market is narrower than the overall gym market, it is still large enough to support pilot deployments and scaling.

\paragraph{Economic Viability — \textit{Mid}.}
Potential revenue streams include system sales, installation fees, and maintenance contracts. Sustainability grants may support early pilots. Although the mechanical system has relatively low wear, the required safety measures and custom installations increase initial costs. The economics appear viable but require carefully chosen pilot sites.

\subsubsection{Challenge}

\paragraph{Implementation Obstacles — \textit{High}.}
The main barriers are physical: ceilings, floor anchoring, retrofitting difficulties, and safety regulations. Each installation needs some customization. Meeting certification requirements for public, high–traffic spaces is also a challenge.

\paragraph{Time to Revenue — \textit{Mid--High}.}
Developing, testing, and certifying the system will require several iterations. Gym procurement cycles are typically slow, especially for equipment that requires installation work. A pilot–driven approach helps but does not eliminate the relatively long time to meaningful revenue.

\paragraph{External Risks — \textit{Mid}.}
There are competing green–gym technologies like regenerative bikes or energy–harvesting floors. Supply chain risks exist for mechanical parts, and insurance requirements might add constraints. There is also a perception risk: if energy output is marketed too strongly, users may feel misled. Clear communication is essential.

\subsubsection{Summary Ratings}

\begin{itemize}
  \item Potential: \textbf{High}
  \item Challenge: \textbf{Mid--High}
  \item Economic viability: \textbf{Mid}
  \item Overall attractiveness: \textbf{Promising for targeted pilots}
\end{itemize}

\subsection{Customer Discovery: Interview Insights}

We conducted an initial gym–user interview and compared the responses to general observations about gym behaviour. Several insights were repeatedly confirmed:

\begin{enumerate}
  \item Users struggle to see progress during workouts without tracking apps.
  \item Many people find gym routines repetitive; anything that adds novelty helps motivation.
  \item Gym loyalty is low—people switch gyms for better equipment or experience.
  \item Sustainability matters, but only if it enhances the workout experience.
  \item The gravity system idea was received positively and described as “cool” or “interesting,” but users were realistic about the low energy output.
\end{enumerate}

Sample interview quotes include:
\begin{itemize}
  \item ``Most machines don't give you a real sense of progress. You just hope you're improving.''
  \item ``I get bored easily. Anything interactive or visual would keep me engaged longer.''
  \item ``If a gym had something unique like this, I'd try it for sure.''
\end{itemize}

\subsection{SWOT Analysis}

\textbf{Strengths}
\begin{itemize}
  \item Very strong visual and motivational impact.
  \item Clear sustainability value for marketing and branding.
  \item Simple mechanical system with relatively low wear.
\end{itemize}

\textbf{Weaknesses}
\begin{itemize}
  \item Low energy output.
  \item Significant installation and space requirements.
  \item Higher responsibility due to safety compliance.
\end{itemize}

\textbf{Opportunities}
\begin{itemize}
  \item University and research pilots.
  \item Boutique gyms looking for unique concepts.
  \item Sustainability funding opportunities.
\end{itemize}

\textbf{Threats}
\begin{itemize}
  \item Competing green fitness technologies.
  \item Negative perception if marketed primarily as an energy–saving device.
  \item Installation delays or supply chain issues.
\end{itemize}

\subsection{Interview Plan}

To strengthen our evidence base, we plan to:
\begin{itemize}
  \item Conduct 8–12 additional user interviews across different gym types.
  \item Interview couple gym managers or owners to understand purchasing behaviour and constraints.
  \item Launch a short online survey for members of the pilot gym.
\end{itemize}

\subsection{Risks and Mitigations}

\paragraph{Low perceived energy yield.}  
We will frame the system as an engagement tool first, with energy production as a bonus.

\paragraph{Space and retrofit issues.}  
We will target early adopters with available space and design modular installation options.

\paragraph{Safety concerns.}  
The system will include multiple safety layers and will undergo independent certification.

\subsection{Conclusion}

The gym and fitness market presents a promising opportunity for the gravity–based system. While the electrical output remains modest, the system's impact on engagement and sustainability storytelling gives it a strong value proposition. A focused pilot strategy will allow us to validate user motivation, refine the technical design, and assess willingness to pay. If pilot results are positive, the opportunity can be scaled into niche but meaningful segments of the fitness market.

    %%%%%%%%%%%%%%%%%%%%%%%%%%%%%%%%%%%%%%%%%%%%%%%%%%%%%%%%%%%%%
    %%%%%%%%%%%%%%%   PART 2 — EDUCATION KITS   %%%%%%%%%%%%%%%%%
    %%%%%%%%%%%%%%%%%%%%%%%%%%%%%%%%%%%%%%%%%%%%%%%%%%%%%%%%%%%%%
    
    \newpage
    \section*{Market Opportunity Assessment: Gravity-Based Energy Storage (GBES) Education Kits}

        \noindent
        \textbf{Target Segment:} High school physics laboratories and university civil/mechanical engineering departments seeking safe, hands-on methods to teach core energy storage and mechanics concepts.

    \section{Potential}
    
        \subsection{Compelling Reason to Buy (High)}
    
            \subsubsection*{Unmet Need: Safety in Energy Storage Demonstrations}
                Educational institutions face what may be called a ``Safety Paradox'': curricula increasingly require students to understand energy storage systems, yet commonly available technologies such as lithium-ion batteries introduce non-trivial risks, including thermal runaway and fire hazards (NFPA, 2024). Many school districts mandate that chemical batteries be stored in certified safety containers, which prevents meaningful hands-on engagement. The GBES kit directly addresses this unmet need by offering a non-chemical, non-flammable storage mechanism suitable for everyday classroom use.
    
            \subsubsection*{Effective Solution: Making Energy Visible}
                Traditional electrochemical storage devices hide the physical mechanism of energy retention; charged and discharged states appear identical, creating the ``Invisible Energy Problem.'' In contrast, GBES exemplifies the potential energy relationship (PE = mgh) through visible, tangible changes. Students can watch weights being raised and lowered, turning energy storage into an intuitive and kinetic process. This aligns with the growing emphasis on experiential learning, a primary driver of science lab equipment adoption (Dataintelo, 2024).
            
            \subsubsection*{Better Than Current Solutions}
                Most renewable energy educational kits emphasize generation (solar, wind) and rely on chemical batteries or capacitors for storage. These introduce safety, handling, and disposal concerns. By eliminating thermal runaway risks, GBES enables unrestricted hands-on experimentation. This makes it a uniquely differentiated product in the current STEM education market.
    
        \subsection{Market Volume (Moderate--High)}
    
            \subsubsection*{Current Market Size}
                The global Science Lab Equipment for Education market reached \$2.84 billion in 2024 (Dataintelo, 2024). The broader K--12 STEM market was valued at \$49.6 billion in 2023 (Market.us, 2024), illustrating substantial purchasing power among the target audience.
    
            \subsubsection*{Expected Growth}
                The STEM education sector is projected to expand to \$177.5 billion by 2033, with a CAGR of 13.6\% (Market.us, 2024). This growth is propelled by modernization initiatives, engineering-focused curricula, and the adoption of Next Generation Science Standards (NGSS). These trends indicate strong demand for innovative instructional systems such as GBES.
    
        \subsection{Economic Viability (High)}
            
            \subsubsection*{Margins (Value vs. Cost)}
                Comparable physics laboratory kits command significant prices. PASCO Scientific’s Mechanics Starter Kit sells for \$789 (PASCO, 2025), while Horizon Educational’s renewable energy kit retails for about \$434 (Toolkit Technologies, 2024). GBES, using low-cost mechanical components (weights, pulleys, small generator), could be priced at \$250--\$300, undercutting competitors by 30--60\% while maintaining strong gross margins.
            
            \subsubsection*{Value--Cost Proposition}
                Schools with limited budgets routinely prioritize durable, low-maintenance equipment. GBES components have long operational lifespans, no consumables, and no chemical waste. This makes GBES an attractive alternative to battery- and fuel-cell-based solutions.
    
    \section{Challenge}
    
        \subsection{Implementation Obstacles (High)}
    
            \subsubsection*{Product Development Difficulties}
                Transitioning from prototype to classroom-ready product involves extensive design-for-manufacture (DFM) refinement. Child-safe, impact-resistant plastic components require injection molding processes that typically take 9--12 months. This adds substantial development time prior to commercialization.
    
            \subsubsection*{Sales and Distribution Difficulties}
                The most significant barrier is the educational ``procurement labyrinth.'' Purchases often require district-level approval, vendor registration, and compliance with numerous administrative requirements. Studies show that over 40\% of superintendents authorize fewer than 15 new product purchases annually, largely due to regulatory and budgetary constraints (AASA, 2024).
    
            \subsubsection*{Funding Challenges}
                Following the expiration of ESSER federal relief funds in September 2024, schools entered a ``Fiscal Cliff'' (GFOA, 2024). Many districts redirected budgets toward essential staffing and maintenance needs, freezing or delaying investment in new hardware such as laboratory kits.

    \subsection{Time to Revenue (High)}

        \subsubsection*{Development Time}
            Schools tend to be risk-averse and seldom adopt new equipment without evidence of successful use in peer institutions. A pilot period of at least six months is often required to gather testimonials and build trust among early adopters.

        \subsubsection*{Time Between Product and Market Readiness}
            Achieving a manufacturing cost below \$50 per unit requires multiple DFM iterations. These adjustments extend the commercialization timeline, pushing revenue realization farther into the future.

        \subsubsection*{Length of Sales Cycle}
            Academic procurement cycles operate within fiscal-year budgets (July--June). This creates sales cycles lasting 6--18 months. Missing the March--May ``Spring Buying Season'' can defer purchase decisions for an entire year.
            
    \subsection{External Risks (Moderate--High)}
        
        \subsubsection*{Competitive Threat}
            The virtual laboratory market reached \$1.8 billion in 2024, with rapid 17.8\% CAGR growth (MarketIntelo, 2024). VR/AR platforms offer zero- maintenance alternatives to physical kits and integrate seamlessly with Chromebook/iPad-based classrooms. These digital substitutes appeal to budget-constrained districts seeking scalable, low-overhead options.
        
        \subsubsection*{Third-Party Dependencies}
            Affordable manufacturing relies on overseas injection molding. Disruptions such as tariffs, shipping delays, or geopolitical tensions could jeopardize timely delivery, especially before critical academic deadlines like the Back-to-School period.
        
        \subsubsection*{Barriers to Adoption}
            School districts often require vendors to provide insurance documentation, agree to extended payment terms (Net-30/60), and comply with stringent procurement procedures. These demands pose substantial challenges for early-stage startups without dedicated administrative staff.
        
    \section*{References}
        
        AASA (2024). \textit{Top 10 Problems on the Job}. The School Superintendents Association. \\
        Dataintelo (2024). \textit{Science Lab Equipment For Education Market Research Report 2033}. \\
        GFOA (2024). \textit{The End of ESSER: The Fiscal Cliff for School Districts}. Government Finance Officers Association. \\
        Market.us (2024). \textit{STEM Education In K--12 Market Size, Share}. \\
        MarketIntelo (2024). \textit{Virtual Labs Market Research Report 2033}. \\
        NFPA (2024). \textit{Lithium-Ion Battery Safety}. National Fire Protection Association. \\
        PASCO Scientific (2025). \textit{Mechanics Starter Kit (ME-5300)}. \\
        Toolkit Technologies (2024). \textit{Horizon Renewable Energy Science Kit 2.0}. \\

    
    
    %%%%%%%%%%%%%%%%%%%%%%%%%%%%%%%%%%%%%%%%%%%%%%%%%%%%%%%%%%%%%
%%%%%%%%%%%%%%%   PART 3 — SOLAR & WIND ENERGY   %%%%%%%%%%%
%%%%%%%%%%%%%%%%%%%%%%%%%%%%%%%%%%%%%%%%%%%%%%%%%%%%%%%%%%%%%

\section{Gravity-Based Energy Storage for Solar \& Wind Energy Farms}

With an increased demand for renewable energy sources such as wind and solar, effective energy storage is essential due to their dependence on weather conditions. Without storage, excess energy could be wasted and 24/7 energy availability cannot be guaranteed.

Currently, solar and wind farms commonly use:

\begin{itemize}
    \item \textbf{Lithium-Ion Batteries:} High energy density, long cycle life, fast charging, and compact installation. Drawbacks include high cost, limited lifespan, safety concerns, environmental impact, and temperature sensitivity.
    \item \textbf{Thermal Energy Storage (TES):} Stores excess energy as heat in materials, making use of otherwise wasted energy. Benefits include high energy density; drawbacks are safety risks, environmental sensitivity, and heat losses during storage and use.
\end{itemize}

\textbf{Gravity-Based Energy Storage (GBES)} offers several advantages over these methods:

\begin{itemize}
    \item Minimal environmental limitations and dependency on weather.
    \item Safer and environmentally friendly, with no rare minerals or dangerous chemicals.
    \item Low energy loss during storage, as it does not dissipate as heat.
\end{itemize}

Large-scale GBES facilities, such as the Energy Vault project in Rudong, China, demonstrate practical implementation. Target customers include municipalities and corporations managing large-scale renewable energy farms.

This application has been analysed using MON Worksheet 2. 

% Insert your image here
\begin{figure}[h!]
    \centering
    \includegraphics[width=0.8\textwidth]{assets/mon-2.png} 
    \caption{Example of a gravity-based energy storage facility for renewable energy.}
    \label{fig:GBES}
\end{figure}

    %%%%%%%%%%%%%%%%%%%%%%%%%%%%%%%%%%%%%%%%%%%%%%%%%%%%%%%%%%%%%
    %%%%%%%%%%%%%%%           APPENDIX          %%%%%%%%%%%%%%%%%
    %%%%%%%%%%%%%%%%%%%%%%%%%%%%%%%%%%%%%%%%%%%%%%%%%%%%%%%%%%%%%
    
    \section{Appendix}
        \input{sections/appendix/copilot-responds}
        
\end{document}
